\section{M. 차원문}

\begin{frame} % No title at first slide
    \sectiontitle{M}{차원문}
    \sectionmeta{
        \texttt{permutation\_cycle\_decomposition}\\
        출제진 의도 -- \textbf{\color{acgold}Challenging}
    }
    \begin{itemize}
        \item 제출 6번, 정답자 0명 (정답률 0.000\%)
        \item 처음 푼 사람: \textbf{없음}
        \item 출제자: \texttt{kaorin}
    \end{itemize}
\end{frame}

\begin{frame}{\textbf{M}. 차원문}
    \begin{itemize}
        \item $i$에서 차원문을 사용해 $a_i$로 이동하는 과정을 $f(i)$라고 합시다. 이걸 $k$번 반복해서 얻은 값을 $f^k(i)$라고 놓는다면, $f^k(i)=i$가 되는 어떤 값 $k$가 존재합니다. 이 과정에서 나온 값의 집합 $f(i),f^2(i),f^3(i),\cdots,f^k(i)$은 사이클을 이루게 되므로 이 문제가 순열 사이클 분할 문제임을 알 수 있습니다.
        \item 결국 이 문제는 순열의 원소들을 적절하게 뒤바꾸면서 $f^k(i)=i$를 만족시키는 최소의 양의 정수 $k$가 $N$이 되도록 만드는 문제입니다.  
    \end{itemize}
\end{frame}

\begin{frame}{\textbf{M}. 차원문}
    이 문제를 해결하기 위해서 아래와 같은 관찰이 필요합니다.
    \vspace{16pt}
    \begin{itemize}
        \item 서로 다른 두 사이클의 원소를 교환하면 두 개의 사이클이 하나로 합쳐지면서 사이클의 개수가 $1$ 줄어듭니다.
        \item 어떤 사이클의 크기가 $N$이 아니라면, 사이클 안에 있는 어떤 원소에 대해서 크기 차이가 $1$인 원소가 다른 사이클에 반드시 존재합니다.
    \end{itemize}
\end{frame}

\begin{frame}{\textbf{M}. 차원문}
    \begin{itemize}
        \item 초기 순열 사이클의 개수가 $C$라고 할 때, 위 관찰을 이용해서 답을 구해보면 사이클들을 하나로 합치는 과정은 $C-1$이며, 이러한 과정을 거칠때 필요한 마나의 비용은 $1^2=1$이 되어 $C-1$이 됩니다.
        \item 위 내용을 이해했다면 해를 구성하는 것 역시 가능합니다. 순열 사이클 분할을 한 후에 아무 원소나 하나 골라서 차이가 $1$인 원소가 다른 사이클에 존재한다면 그 두개의 사이클을 하나로 합쳐주는 과정을 반복하면 됩니다. 이 과정의 시간복잡도는 $O(N)$입니다.
    \end{itemize}
\end{frame}

