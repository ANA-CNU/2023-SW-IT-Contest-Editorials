\section{J. 전구 상태 바꾸기}

\begin{frame} % No title at first slide
    \sectiontitle{J}{전구 상태 바꾸기}
    \sectionmeta{
        \texttt{greedy}\\
        출제진 의도 -- \textbf{\color{acgold}Hard}
    }
    \begin{itemize}
        \item 제출 27번, 정답자 4명 (정답률 14.814\%)
        \item 처음 푼 사람: \textbf{배민수}, 94분
        \item 출제자: \texttt{ygonepiece}
    \end{itemize}
\end{frame}

\begin{frame}{\textbf{J}. 전구 상태 바꾸기}
    \begin{itemize}
        \item 연속한 세 전구의 상태를 바꾸는 연산은 연산의 순서가 중요하지 않습니다. 어떠한 순서로 연산을 적용하건 최종적인 전구의 상태는 똑같기 때문입니다.
        \item 연속한 세 전구의 상태는 세 번 이상 바꿀 필요가 없습니다. 세 번 바꾸면 원래 상태로 돌아오기 때문입니다.
        \item 그렇다면 $1, 2, 3$번째 전구의 상태를 몇 번 바꿀지 고민하는 것을 시작으로 $2, 3, 4$번째 전구, $3, 4, 5$번째 전구$\cdots$ 를 계속해서 고려해줄 수 있습니다. 그러나, 각 경우마다 가능한 선택지는 $3$가지이므로 단순하게 생각하면 \complexity{3^N}가지 경우를 모두 시도해봐야 합니다.
    \end{itemize}
\end{frame}

\begin{frame}{\textbf{J}. 전구 상태 바꾸기}
    \begin{itemize}
        \item 전구의 색 모두 빨간색으로 만들어야 한다고 가정하고 이전 과정을 진행해봅시다. $1$번째 전구를 빨간색으로 만들려면 $1, 2, 3$번째 전구의 상태를 바꿔야 하는지는 바로 알 수 있습니다.
        \item 그렇게 $1$번째 전구를 빨간색으로 만들고 나면, $2$번째 전구도 빨간색으로 만들려면 $2, 3, 4$번째 전구의 상태를 몇 번 바꿔야 하는지도 바로 알 수 있습니다.
        \item 그렇게 모든 전구를 빨간색으로 바꾸려면 몇 번 상태를 바꿔야 하는지 알 수 있습니다. 물론 모든 전구가 빨간색으로 빛나게 할 수 없을 수도 있습니다.
        \item 이 과정을 빨간색, 초록색, 파란색에 대해서 모두 시도한 후에 그 중 최솟값을 구하면 문제의 정답입니다.
    \end{itemize}
\end{frame}