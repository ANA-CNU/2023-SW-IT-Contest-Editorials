\section{P. 도미노 수열}

\begin{frame} % No title at first slide
    \sectiontitle{P}{도미노 수열}
    \sectionmeta{
        \texttt{lazy\_segment\_tree}\\
        출제진 의도 -- \textbf{\color{acdiamond}Challenging}
    }
    \begin{itemize}
        \item 제출 60번, 정답자 0명 (정답률 0.000\%)
        \item 처음 푼 사람: \textbf{없음},
        \item 출제자: \texttt{ygonepiece}
    \end{itemize}
\end{frame}

\begin{frame}{\textbf{P}. 도미노 수열}
    \begin{itemize}
        \item 수열의 첫 번째 수부터 넣으면서 도미노 수열을 만들어 봅시다.
        \item 수를 넣을 때, 새로운 도미노 수열을 만들 수도 있고 기존의 것에 수를 이어 붙일 수도 있기 때문에 경우의 수가 많이 생기게 됩니다.
        \item 도미노 수열의 합과 길이를 중심으로 생각해 봅시다.
        \item 만들어진 도미노 수열에 특정 수를 뒤에 붙일 수 있으면 무조건 붙이는 게 이득이라는 것을 알 수 있습니다.
    \end{itemize}
\end{frame}

\begin{frame}{\textbf{P}. 도미노 수열}
    \begin{itemize}
        
        \item 이를 토대로 Naive한 DP식을 떠올려 보겠습니다.
        \item $a_i$ 이상의 합을 이루는 도미노 수열 뒤에 $a_i$를 붙이고, 도미노 수열의 합이 $w$을 이루는 수열 중 최대 길이를 $dp[i][w]$라고 합시다.
        \item 초기 점화식은 $dp[i][a_i] = 1$이고, 점화식은 $dp[i][w] = dp[i-1][w-a_i] + 1 \ (a_i \leq w-a_i) \ (1 \le dp[i-1][w-a_i])$, $dp[i][w] = dp[i-1][w] \ (w < a_i)$입니다.
        \item 만약 $dp[i-1][w-a_i]$의 값이 0이라면 합이 $w - a_i$인 도미노 수열이 존재 하지 않으므로 $a_i$를 덧붙일 수 없습니다.
    \end{itemize}
\end{frame}

\begin{frame}{\textbf{P}. 도미노 수열}
    \begin{itemize}
        \item 위 점화식은 $a_i$이상의 무게를 갖는 모든 도미노 수열에서 $a_i$가 덧붙여졌고 길이도 1 증가 한다는 특징을 가집니다.
        \item $w$가 굉장히 클 수 있기 때문에 모든 $w$를 Naive하게 탐색하며 dp식을 적용 할 수 없습니다.
        \item 존재하는 $w$만 관리하여 $a_i$를 덧붙이면 한번의 작업에 $O(N)$의 시간이 걸리게 됩니다.
    \end{itemize}
\end{frame}

\begin{frame}{\textbf{P}. 도미노 수열}
    \begin{itemize}
        \item 레이지 세그먼트 트리를 이용하면 빠른 시간안에 도미노 수열의 합이 $a_i$이상인 모든 도미노 수열의 길이을 $1$ 증가 시키고 무게도 $a_i$ 증가 시킬 수 있습니다.
        \item $dp[i][w]$가 $1$ 이상인 모든 $w$는 $i+1$번째 연산 후, $w+a_{i+1} (a_{i+1} \leq w)$ 으로 값이 $1$ 더해져 전이하게 됩니다.
        \item 즉, 서로 다른 도미노 수열의 합의 크기 순서는 변하지 않습니다.
    \end{itemize}
\end{frame}

\begin{frame}{\textbf{P}. 도미노 수열}
    \begin{itemize}
        \item $a_i$보다 합이 큰 도미노 수열이 존재하지 않는 경우에는 새로운 도미노 수열을 만들면 됩니다.
        \item $a_i$보다 합이 큰 도미노 수열이 존재한다면, $a_i$로 시작하는 도미노 수열을 만드는 것보다 그 수열에 덧붙이는 게 이득이기 때문입니다.
        \item 따라서 이 같은 경우는 가장 긴 도미노 수열이 될 수 없기 때문에, $a_i$로 시작하는 도미노 수열은 만들지 않습니다. 
    \end{itemize}
\end{frame}

\begin{frame}{\textbf{P}. 도미노 수열}
    \begin{itemize}
        \item 필요 없는 도미노 수열은 만들지 않으면서 도미노 수열들을 관리하게 되면 무게를 기준으로 정렬된 도미노 수열들을 관리 할 수 있습니다.
        \item 그러면 각 도미노 수열을 특정 인덱스로 잡고, 무게를 관리하는 레이지 세그먼트 트리와 길이를 관리하는 레이지 세그먼트 트리로 관리 할 수 있습니다.
        \item 무게를 관리하는 레이지 세그먼트 트리를 이용해서 덧붙일 수 있는 도미노 수열들이 있는 구간을 찾고, 레이지하게 무게와 길이를 업데이트 하면 됩니다.
        \item 이때, 길이는 실시간으로 파악하지 않아도 되므로 업데이트에 $O(1)$ 마지막에 $O(N)$으로 관리해줘도 됩니다.
    \end{itemize}
\end{frame}

\begin{frame}{\textbf{P}. 도미노 수열}
    \begin{itemize}
        \item 도미노 수열들의 무게를 관리하고 무게로 이분 탐색을 하기 위해서는 구간 최댓값 세그먼트 트리를 이용하면 됩니다.
        \item 세그먼트 트리 왼쪽으로 우선 내려가며, 인자로 받은 무게 값보다 구간 최댓값이 더 작다면 리턴하여 올라오면 됩니다.
        \item 따라서 이 문제는 $O(N\log N)$으로 해결 할 수 있습니다.
    \end{itemize}
\end{frame}