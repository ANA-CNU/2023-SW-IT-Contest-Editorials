\section{B. 치즈버거 만들기}

\begin{frame} % No title at first slide
    \sectiontitle{B}{치즈버거 만들기}
    \sectionmeta{
        \texttt{math}\\
        출제진 의도 -- \textbf{\color{acbronze}Easy}
    }
    \begin{itemize}
        \item 제출 139번, 정답자 61명 (정답률 43.884\%)
        \item 처음 푼 사람: \textbf{미미미누}, 1분
        \item 출제자: \texttt{ygonepiece}
    \end{itemize}
\end{frame}

\begin{frame}{\textbf{B}. 치즈버거 만들기}
    \begin{itemize}
        \item 만약 패티가 치즈보다 많다면 만들 수 있는 버거의 최대 크기는 $2 \times B+1$입니다.
        \item 그렇지 않을 경우에는 $2 \times A-1$입니다.
        \item 위와 같이 경우의 수를 나누지 않고 $2 \times \min(A-1,B) + 1$으로 정답을 구할 수도 있습니다.
        \item 수식으로 계산하지 않고, 패티와 치즈가 남아있을 때까지 치즈버거에 계속 쌓는 과정을 시뮬레이션해서 구할 수도 있습니다.
    \end{itemize}
\end{frame}
