\section{L. K의 배수}

\begin{frame} % No title at first slide
    \sectiontitle{L}{K의 배수}
    \sectionmeta{
        \texttt{dynamic\_programming}\\
        출제진 의도 -- \textbf{\color{acgold}Hard}
    }
    \begin{itemize}
        \item 제출 11번, 정답자 2명 (정답률 18.182\%)
        \item 처음 푼 사람: \textbf{박현민}, 58분
        \item 출제자: \texttt{kangwlgns}
    \end{itemize}
\end{frame}

\begin{frame}{\textbf{L}. K의 배수}
    \begin{itemize}
        \item $j-1$자리 수인 어떤 수 $p$의 맨 뒤에 숫자 하나를 더 붙여 $j$자리 수를 만드는 과정을 $p \times 10+q (0 \leq q \leq 9)$로 표현할 수 있습니다.
        \item 예를 들어 $13$을 세 자리로 확장하면 앞의 두 자리가 $13$인 세 자릿수가 나오게 됩니다. 그런 수는 $130, 131, \cdots, 139$가 있고, 이는 $13 \times 10 + q$로 표현할 수 있습니다.
        \item 자릿수를 확장하는 과정에서 곱셈과 덧셈만 사용됐으므로 $\mod$연산의 성질에 의해서 다음이 성립합니다.$(p \times 10 + q) \mod K = (((p \mod K) \times 10) \mod K + q \mod K) \mod K$.
    \end{itemize}
\end{frame}

\begin{frame}{\textbf{L}. K의 배수}
    \begin{itemize}
        \item $j$ 자릿수 중 $K$로 나눈 나머지가 $i$인 수의 개수를 점화식 $R(i, j)$라고 할 때, 위처럼 모듈러 연산을 이용하면 $R(i, j-1)$로부터 $R(((i \mod K) \times 10 \mod K + a_n \mod K) \mod K, j)$를 도출할 수 있습니다.
        \item $10 \mod K$는 간단하게 구할 수 있고, $q \mod K$는 사용 가능한 숫자 $a_n (1 \leq n \leq N; 0 \leq a_n \leq 9)$들을 $K$로 나눈 나머지를 의미하므로 간단하게 구할 수 있습니다.
        \item $p \mod K$를 매번 구한다면 오래 걸리므로 다이나믹 프로그래밍을 이용하여 시간복잡도를 줄일 수 있습니다.
    \end{itemize}
\end{frame}

\begin{frame}{\textbf{L}. K의 배수}
    \begin{itemize}
        \item 점화식을 이용해 DP테이블을 채워 나가면 $R(0,M)$를 얻을 수 있고, $K$의 배수는 $K$로 나눈 나머지가 $0$이라는 것과 같으므로 $R(0,M)$를 출력하면 정답입니다.
        \item $M$ 자릿수이고 $K$로 나눈 나머지는 $K$개 존재하므로 DP테이블을 생성하기 위해  $O(NM)$의 공간이 필요하고, 매 테이블을 채우는 과정에서 $N$개의 숫자에 대해 모듈러 연산을 하였으므로 시간복잡도는 $O(NMK)$입니다.
        \item 첫 자리가 $0$이 아니어야 했으므로 $j = 0$일 때 $0$을 세지 않음에 주의합니다.
    \end{itemize}
\end{frame}
