\section{N. 비밀의 화원}

\begin{frame} % No title at first slide
    \sectiontitle{N}{비밀의 화원}
    \sectionmeta{
        \texttt{parametric\_search, sweeping}\\
        출제진 의도 -- \textbf{\color{acplatinum}Challenging}
    }
    \begin{itemize}
        \item 제출 12번, 정답자 0명 (정답률 0.000\%)
        \item 처음 푼 사람: \textbf{없음}
        \item 출제자: \texttt{kaorin}
    \end{itemize}
\end{frame}

\begin{frame}{\textbf{N}. 비밀의 화원}
    \begin{itemize}
        \item 장애물이 없고 $N$과 $M$이 커진 토마토 문제입니다. 토마토와 같은 풀이를 이용하면 메모리 초과가 발생합니다.
        \item 풀이를 바꿔봅시다. 화원의 모든 칸에 꽃이 피워진다는 것은 각각의 행들 모두 모든 칸에 꽃이 피워져 있다는 것과 같습니다.
        \item 따라서 $N$과 $K$가 상대적으로 작다는 것을 이용해 시간을 변수로 잡고 처음에 심었던 꽃이 각 행마다 얼마나 피웠는지를 저장한 후에, 각 행마다 꽃이 피워진 구간들을 전부 합쳐봤을 때 모든 칸이 채워져있는지를 확인하면 됩니다.
    \end{itemize}
\end{frame}

\begin{frame}{\textbf{N}. 비밀의 화원}
    \begin{itemize}
        \item 특정 시간 $t$에서 화원의 모든 칸에 꽃을 피웠다면 $1$이상의 값 $k$에 대해서 $t+k$시간에도 화원의 모든 칸에는 꽃이 피워져 있을 것이고, $t$시간 미만이라면 화원의 모든 칸에 꽃이 피워져 있지 않을 것입니다.
        \item 즉 매개 변수 탐색을 이용하여 $t$를 빠르게 탐색할 수 있습니다.
    \end{itemize}
\end{frame}

\begin{frame}{\textbf{N}. 비밀의 화원}
    \begin{itemize}
        \item 구간을 합치는 과정은 스위핑으로 해결이 가능합니다. 
        \item 각 행에 대해서 $K$개의 꽃들이 피워진 구간들을 저장한 후에 정렬 과정을 거쳐서 붙이면 $O(K\log K)$시간에 그 행의 모든 칸에 꽃이 피워져 있는지 판별이 가능합니다.
        \item $N$개의 행에 걸쳐서 $K$개의 꽃들이 피워진 구간을 정렬하므로 시간복잡도는 $O(NK\log K)$입니다.
        \item 최종 시간복잡도는 $O(NK\log K \log(N+M))$이 됩니다.
        \item 하지만 시간초과가 날 확률이 높습니다.
    \end{itemize}
\end{frame}

\begin{frame}{\textbf{N}. 비밀의 화원}
    매개 변수 탐색 이전에 x축을 기준으로 정렬해두는 것도 가능합니다. 
    구간의 가운데를 $X$, 구간의 왼쪽 끝을 $L$, 구간의 오른쪽 끝을 $R$이라고 놓는다면 정렬했을 경우 다음 조건들이 성립합니다.
    \vspace{18pt}
    \begin{itemize}
        \item $X_i<X_j$면서 $L_i>L_j$라면 구간 $i$는 구간 $j$에 포함되어 있다.
        \item $X_i<X_j$면서 $R_i>R_j$라면 구간 $j$는 구간 $i$에 포함되어 있다.
    \end{itemize}
\end{frame}

\begin{frame}{\textbf{N}. 비밀의 화원}
    \begin{itemize}
        \item 이를 이용해서 다른 원소에 포함되는 경우들을 제외하고 남은 구간들은 $L$값이 작은 순으로 정렬되어 있게 됩니다. 
        \item 이제 이 구간들을 전부 합쳤을 때, 모든 칸이 채워지는지 확인해주면 됩니다.
        \item 이때 시간복잡도는 $O(NK \log(N+M)+ K\log K)$가 되며, 위의 방식보다 약 10배 정도 빠르게 해결할 수 있습니다.
        \item 여담으로, $O(NK)$에 문제를 해결하는 방법도 있습니다.
    \end{itemize}
\end{frame}
